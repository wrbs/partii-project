\documentclass[12pt,a4paper, headings=standardclasses]{scrartcl}

\usepackage[T1]{fontenc}
\usepackage[utf8]{inputenc}

\usepackage{kpfonts}
\usepackage{hyperref}
\usepackage{booktabs}
\usepackage{tabularx}

\newcommand{\tabitem}{~~\llap{\textbullet}~~}

\begin{document}

\title{A JIT compiler for OCaml bytecode}
\author{William Robson (wnr21), Selwyn College}
\date{23 October 2020}

\maketitle

\begin{center}
    \begin{tabular}{rl}
        \textbf{Special Resources:}   & \textit{None required}           \\
        \textbf{Project Supervisor:}  & Timothy Jones                    \\
        \textbf{Director of Studies:} & Richard Watts                    \\
        \textbf{Project Overseers:}   & Neel Krishnaswami \& Sean Holden \\
    \end{tabular}
\end{center}

\vspace{2cm}

\section*{Introduction}

The OCaml compiler has two main targets - to platform-specific machine code
and an interpreted bytecode.

Although the bytecode interpreter is fairly well optimised for interpreting a
functional language it still has some degree of overhead over what even a
simple JIT compiler could achieve as it cannot benefit from modern CPU
pipelining and branch prediction as an abstract machine PC is used rather than
the CPU's.

This project aims to experiment with replacing the core interpreter loop with
JIT compiling the bytecode and to quantify how much faster could be achieved.

Beyond this basic core the project aims to use remaining time to discover and
implement performance optimisations in the compiler and its emitted machine
code.

To make the project more interesting and allow the use of higher level
features while writing the interpreter the project will have as much of it as
possible written in the Rust programming language. Both the compiled native
code and bytecode of the existing compiler have a runtime component written
in C. Therefore the project will also serve as an experiment into what extent
it is possible to create a safer abstraction around the OCaml runtime.

\subsection*{Technical Decisions and Scope}

The project will only target x86\_64 Linux and macOS and no attempts will be made
two be portable to other platforms.

The bytecode interpreter implements an abstract machine which uses a shared
stack heavily. My preliminary research has found that the design of the
abstract machine (especially with regards to functional language features
like partial application of functions) makes it very difficult to modify the
calling convention or structure of the bytecode.

The way closures are implemented in this abstract machine means that the common JIT
design of delaying the compilation of a block of code until it is determined
to be in the hot path is more difficult to implement correctly.
\footnote{Functions do not have a fixed entry and exit point as the abstract
    machine follows the push-enter not eval-apply model described in slides
    32-43 of Leroy, KAZAM 2005
    \href{https://xavierleroy.org/talks/zam-kazam05.pdf}{https://xavierleroy.org/talks/zam-kazam05.pdf}}

Therefore initial design of the JIT compiler will be to compile
all bytecode sections as they are loaded (at program startup or by the
toplevel) to machine code, which is then branched to. This entirely replaces the
interpreter loop rather than augmenting it. In all other aspects the compiled
code will behave identically to the interpreter but with set concrete machine
registers for the abstract machine registers.

Most of the work doing all compilation upfront could be reused mostly without change
in a future more elaborate scheme as a potential extension.

\section*{Work that has to be done}

\subsection*{Starting Point}

I have decent working knowledge of Rust, C and OCaml. I have experience with
writing virtual machines and emulators but I have not yet learned as much about
the x86\_64 platform.

This project forks from the OCaml compiler source 4.11.1 \footnote{Latest at
    time of project start}. For this reason all work is licensed under the same
LGPL 2.1 with linking exception.

In preparation for this project prior to formal start I have already:

\begin{itemize}
    \item Modified the Makefiles and autoconf scripts to support linking to
          a static library component written in Rust.
    \item Added a Rust hook functions when bytecode gets loaded and unloaded
          and the interpreter is called.
    \item Added Rust type definitions for the structure of the bytecode instructions
          and a parser and disassembler for them.
\end{itemize}

To avoid confusion about what work was completed before the project start
submission of the code will have 3 components:

\begin{itemize}
    \item The final state of the code
    \item A full git history \footnote{as a \href{https://git-scm.com/docs/git-archive}{git-archive} file}
    \item A snapshot of the state of the code at time of formal project start for comparisons
\end{itemize}

This issue will also be covered explicitly in the final dissertation to avoid
misleading the examiners about how much was achieved in the time allotted.

However, the work already completed is nearly all tooling and as such does
not overlap with the core work to be completed.

\subsection*{Success criteria}

I will consider the project successful if

\begin{enumerate}
    \item There is a JIT compiler implemented into the existing OCaml source
          replacing the interpreter
    \item There is a useful suite of benchmarks built comparing its
          performance to other alternatives
    \item It performs favourably when discounting initial compile time for all programs
          and overall for longer running programs
\end{enumerate}

In addition to the basic criteria I have left time in the plan for
performance and functionality related extensions.

However, I cannot yet say what these will be as I cannot measure where the overhead lies most.

\subsection*{Components}

This breaks down into these major components:

\begin{enumerate}
    \item Design a scheme for mapping original bytecode locations to absolute
          machine code addresses and use it instead of bytecode-relative instructions
          for the implementation of closures.
    \item A solution for allocating memory, emitting instructions to it,
          relocating labels, marking the region as executable and jumping to it.
    \item The compiler itself, which emits assembly for a given stream of bytecode
          instructions
    \item Adding any new support primitives that are better implemented by functions
          than inlining into the compiler
    \item Collecting a series of test programs and benchmarks and creating
          tooling to automate running them against the native code compiler's
          output and the existing interpreter
\end{enumerate}

In pursuit of this goal I can foresee that tools and tests will be created to
help with debugging or stub out not yet implemented features. It is hard to
say at this point exactly what will become necessary for success but I can
envision now:

\begin{enumerate}
    \item Smaller, self-contained tests of just the compiler's output.
    \item A way of comparing traces of the interpreter and the JIT compiler down to the registers and stack state. \footnote{This may require partially modifying the garbage collector to make when it decides to trigger deterministic}
\end{enumerate}

However, these are not the core goals of the project and instead things that
may be created to make achieving them less painful.

\subsection*{Evaluation}

The key metrics are benchmarks and which features are supported.

For benchmarking I will measure:

\begin{enumerate}
    \item Time to startup and start interpreting\footnote{If I have time due to completing the plan significantly early and change when compilation happens, this will change}
    \item Program execution time
\end{enumerate}

I will compare these values and the total execution time against the native
code compiler and existing interpreter on a variety of test programs of
different sizes and types of computation being done.

There is an inherent tradeoff and the aim of the evaluation section will be
to quantify where it lies for different classes of programs.

\section*{Difficulties to Overcome}

The following main learning tasks will have to be undertaken as part of the
project.

\begin{itemize}

    \item Understand the semantics of the instruction set and enough of the
          runtime that the interpreter interacts with.
    \item Learn enough about x86\_64 assembly to be able to know the right
          instructions to emit to replace the existing C.
\end{itemize}

The main difficulty I envision is with the first of these tasks. There isn't
much complete documentation so a lot of the learning will have to be done by
reading the code. I believe that extending my current interpreter rewrite will
help with this at the start but do not plan to finish it.

There are plenty of resources I have found for the second task and I have
confidence that it will be achievable.

The other major potential difficulty is the risk of hard-to-debug issues
slowing progress. Upfront investment into automated testing will help
significantly.

\section*{Resources}

This project only requires a modern Unix environment with the build
requirements for OCaml and the Rust compiler installed. So far I have tested
this on Linux, Windows Subsystem for Linux and macOS.

I plan to use my own computers. I have a desktop that I recently built with a
10700K CPU and 32GB of RAM. It runs triple-boot Windows, Arch Linux and
macOS. All OSes use seperate SSDs.

I also have a 2018 laptop running Arch Linux which is my secondary
development system and can serve as backup against main system failure.

I am using Git for version control of the dissertation source and source code
and the repository is frequently pushed to GitHub. This is my primary line of
defence against system failure.

I additionally have already set up automated synchronisation between all of
my computers and a server I have. This server runs daily incremental backups
of all data to OneDrive.

The repository will also be automatically backed up with this system.

\section*{Work Plan}

\subsection*{Overview}

The goal is to get something working as quickly as possible to allow for
consistent continual testing of actual programs.

To this end the first milestone will be to implement the compiler entirely by
jumping to functions written in C that implement each of the instructions.
These temporary stub functions will take as parameters the same things they
would read from the bytecode.

The main challenge at this stage will be support for relocations. I can build
on my existing work that knows about which instructions have a label but
I will need to create a method for relocating the emitted machine code too.

At this point the system should support all OCaml bytecode programs but with
machine code responsible for branching rather than using the abstract
machine's PC.

Then the goal is to create and collect a suite of test programs that can
easily be run and later used for benchmarks.

From then on work can continue by iteratively implementing more and more
instructions replacing the function call with actual assembly performing the
same task.

I hope to reach the milestone of most programs working by the end of Michaelmas
and have finished the success criteria by the start of Lent.

The rest of Lent will be spent optimising performance and adding extensions to
increase the functionality and performance of the new system.

Code completion is achieved the end of Lent leaving the Easter holidays for
focusing on the dissertation.

\subsection*{Timetable}

The milestone is listed below each chunk of work

\subsubsection*{Michaelmas Term}

\begin{tabularx}{\textwidth}{rX} \toprule
    Dates & Work to be completed                                                                                               \\ \midrule
    16/10 - 28/10
          & \tabitem Decide on method for emitting bytecode from Rust.                                                         \\
          & \tabitem Create mechanism for doing relocations from initial bytecode address to final machine code address.       \\
          & \textit{Creation of framework for emitting machine code and performing relocations. Test with small example.}      \\

          &                                                                                                                    \\

    29/10 - 11/11
          & \tabitem Create C functions implementing all the opcodes from the existing interpreter source.                     \\
          & \tabitem Create compiler emitting jumps to these functions.                                                        \\
          & \textit{Most programs now run using the native code PC as the actual PC}                                           \\

          &                                                                                                                    \\

    29/10 - 11/11
          & \tabitem Start replacing the generated calls to the temporary functions with the assembly code that replaces them. \\
          & \tabitem Start with the smaller and simpler to implement functions first.                                          \\
          & \textit{Most programs now run with some degree of JIT compilation}                                                 \\

          &                                                                                                                    \\

    12/11 - 25/01
          & \tabitem Implement the remaining operations.                                                                       \\
    \textit{Milestone:}
          & \textit{Success criteria nearly completed barring testing}                                                         \\

          &                                                                                                                    \\

    26/11 - 26/01
          & \tabitem Recover from term.                                                                                        \\
          & \tabitem Fix out any smaller fiddly unimplemented features.                                                        \\
          & \tabitem Try to get the compiler building and then use the
    compiler's test suite to test the correctness of the JIT runtime.                                                          \\
          & \tabitem Create a benchmark suite and tooling to run it automated                                                  \\
          & \textit{Benchmarks working and so success criteria done}                                                           \\
    \bottomrule
\end{tabularx}

\subsubsection*{Lent Term}

\begin{tabularx}{\textwidth}{rX} \toprule
    27/12 - 03/02
     & \tabitem Slack time in case initial Michaelmas plan was overly optimistic.                    \\
     & \tabitem Optimise performance of compiler and emitted assembly.                               \\
     & \tabitem Write the progress report.                                                           \\
     & \tabitem Decide on any other extensions to pick up and cover them in the progress report.     \\
     & \textit{Progress report completed. Plan n place for any remaining extensions.}                \\

     &                                                                                               \\

    4/02 - 17/03
     & \tabitem By this point the project should be mainly interesting extensions and optimisations. \\
     & \tabitem Start collecting together notes and references for the dissertation.                 \\
     & \textit{Code completion of project}                                                           \\
    \bottomrule
\end{tabularx}

\subsubsection*{Easter Term}

\begin{tabularx}{\textwidth}{rX} \toprule

    18/03 - 28/04
     & \tabitem Write the first few drafts of the dissertation. \\
    \textit{Milestone:}
     & \textit{A few rounds of dissertation feedback}           \\

     &                                                          \\

    29/05 - 9/06
     & \tabitem Gather and respond to final feedback.           \\
     & \tabitem Ensure bib file is complete.                    \\
     & \tabitem Proofread.                                      \\
     & \tabitem Final submission.                               \\
    \textit{Milestone:}
     & \textit{Project done}                                    \\

     &                                                          \\

    10/06 - 14/06
     & \tabitem Submit early.                                   \\
    \textit{Milestone:}
     & \textit{Project done (officially)}                       \\
    \bottomrule
\end{tabularx}

\end{document}


