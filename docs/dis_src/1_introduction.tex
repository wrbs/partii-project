% !TeX root = ../dissertation.tex
\chapter{Introduction}

This project introduces two new just-in-time (JIT) compilers to x86\_64 machine code for the OCaml
programming language - one focusing on correctness and compile speed and the other on performance
of the compiled code. These compilers are integrated together dynamically: the first compiler is
used for all code and the second compiler is used for functions detected to be called often.  The
combination of compilers system outperforms
\footnote{It is faster for 32/36 of the tested benchmarks: mean speedup = $1.46\times$, $\sigma =
            0.40$}
the
OCaml bytecode interpreter on the majority of tested benchmarks and is capable of executing every
OCaml program.

In this document, I will refer to the faster but less intelligent compiler as the `first' or
`initial' compiler and the slower but more optimising compiler as the `second' or `optimising'
compiler.

\section{Motivation}

OCaml is a compiled functional programming language with a strong, static type system. The OCaml
compiler has two main backends: an optimised target-specific native-code compiler and a compiler to
bytecode which is later run by the interpreter, \texttt{ocamlrun}.

The default bytecode interpreter in OCaml is significantly slower than the output of the
native-code compiler. Although most users of OCaml use the native-code compiler, the bytecode
compiler is
still useful. One of the largest use cases is it is the only method used by the OCaml `toplevel'
read-evaluate-print-loop (REPL). JIT compilation techniques are particularly well suited to this
use case.

This project demonstrates how JIT compilation techniques can be applied to OCaml while retaining
the bytecode format and semantics. As the `toplevel' can only use
bytecode, the JIT could be lead to better performance in interactive OCaml environments.

Additionally, this project demonstrates two contrasting approaches to writing a JIT compiler in
Rust and shows how they can be integrated them together for performance better than either
individually.

\section{Related work}

This project has been attempted before at least twice: with \textsc{OCamlJit} in 2004 \cite{ocjit1}
and \textsc{OCamlJit2} in 2010 \cite{ocjit2}. On advice of my supervisor I decided not to closely
read these papers until after implementation; my first compiler independently ended up with a very
similar design to that of \textsc{OCamlJit2} and my second was more sophisticated than either
project.

This project fits into the space of JITs more generally as a method-based (as opposed to
trace-based) compiler \cite{pyket}.
\section{Work done}

The project achieved on schedule three core goals as set out in the proposal:

\begin{enumerate}
      \item There is a JIT compiler implemented into the existing OCaml source
            replacing the interpreter with all functionality but debugging
            and introspection.
      \item There is a comprehensive and automated suite of benchmarks built
            comparing its performance to other alternatives.
      \item It performs favourably compared to the original interpreter on benchmark programs
\end{enumerate}

I then implemented the significant extension of the optimising compiler and integrated
it with the initial compiler.

The project is primarily written in the Rust and C programming languages. It replaces the
interpreter component of the existing OCaml runtime.

\section{Summary of results}

The system is capable of correctly compiling and executing all OCaml
bytecode instructions and every program I have tested. The features not supported are
the debugger and backtraces (which are not used by default in the existing runtime). The entire
OCaml compiler test suite passes with the exception of tests of
the debugger and backtraces. The JIT compiler can be used in the OCaml compiler's
self-bootstrapping.

The new JIT-using runtime significantly increases performance for most of the tested programs.
The mean speedup achieved across the benchmark suite of 36 different programs was $1.46 \times$.