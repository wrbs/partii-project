% !TeX root = ../dissertation.tex
\chapter{Conclusions}

This project demonstrates how the use of JIT compilation techniques can beat the performance
of manually optimised interpreters. It contrasts two approaches to JIT compilation, showing how
more sophisticated compilers can be applied only when needed for better performance.

Unlike many Part II compiler projects it covers the complete functionality of a large real-world
programming language with a focus on correctness --- beyond not supporting the debugger and
backtraces, no simplifications are made to the language.

\section{Lessons learned}

I found the automated testing strategy very effective for this project --- especially trace
comparison. Starting with the building this meant I could continuously verify my work. When I had a
bug, I could know exactly which instruction led to the mistake rather than debugging indirect
effects later in execution.

I have gained a lot of OCaml domain knowledge over the course of the project and with hindsight
there are a things I would have done differently. In many cases, I had to completely rewrite a
component once I knew more about how it should work. This is difficult to avoid when working within
a large existing system and can be a useful strategy for exploratory design.

However, I could have done a better job mitigating some of the time this took; towards the start of
the project, there were points where I started with a complicated design only to find I didn't need
the component or had to redesign it. By the end of the project, I always started with the simpler
approach.

\section{Possible extensions}

\subsubsection{More sophisticated analysis of types in the optimising compiler}

Performance could be greatly improved by adding OCaml-specific type information to the optimising
compiler. For example, once a value is statically known to be a floating-point value the compiler
can avoid boxing intermediate values and inline the operations, rather than call out to C
primitives. This would greatly help with some of the worst cases for the system which were all
heavy uses of floating-point numbers. Implementing this would require adding an additional
intermediate representation and dataflow analysis passes in between the basic-block stage and
cranelift IR, but would be fairly simple optimisations once the machinery for this was built.

\subsubsection{Tail-call recursion optimisation within functions}

The current compiler could be extended with a moderate amount of work to lower tail-recursive
functions to loops rather than function calls. This is something the native-code compiler does,
since tail-recursion is a common pattern in OCaml. This does not require any support from cranelift
--- only an analysis pass to detect this pattern.

\subsubsection{Extending cranelift to handle tail calls directly}

Cranelift is a new project and its major user, WASM, has no support for emitting efficient machine
code for tail calls.  For this reason, there is no support within the library for avoiding pushing
another stack frame on tail calls. I had to implement tail calls by use of a wrapper, which hurt
performance.
